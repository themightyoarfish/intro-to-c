\documentclass{article}
\usepackage[utf8]{inputenc}
\title{Einführung in C++ -- Übung 3 \\ Testatgruppe A (Isaak)}
\author{Rasmus Diederichsen \and Simon Kern}
\date{\today}
\usepackage[ngerman]{babel}
\usepackage{microtype,
            textcomp,
            xcolor,
            listings,
            tikz,
            IEEEtrantools,
            array,
            amsmath,
            amssymb,
            graphicx,
            subcaption,
            lmodern}
\usepackage[pdftitle={Übung 3}, 
       pdfauthor={Rasmus Diederichsen}, 
       hyperfootnotes=true,
       colorlinks,
       bookmarksnumbered = true,
       linkcolor = lightgray,
       plainpages = false,
       citecolor = lightgray]{hyperref}
\usepackage[T1]{fontenc}
\usepackage[explicit]{titlesec}
\renewcommand{\rmdefault}{ugm}
\usepackage[urw-garamond]{mathdesign}
\usepackage[all]{hypcap}
\titlespacing{\subsection}{0em}{\baselineskip}{1em}
\titleformat{\subsection}{\normalfont\Large\bfseries}{}{0em}{Aufgabe \arabic{section}.\arabic{subsection} #1}

\renewcommand{\thesection}{}

\lstset{
   basicstyle=\footnotesize\ttfamily,
   breaklines=true,
   commentstyle=\color{blue},
   keywordstyle=\color{purple}\textbf,
   numberstyle=\tiny\color{gray},
   numbers=left,
   language=C,
   stringstyle=\color{olive},
}

\begin{document}

   \maketitle

   \section{}
   \setcounter{section}{3}

   \subsection{Pointerarithmetik}
   \lstinputlisting[language=C,caption={pointer\_test.c}]{src/pointer_test.c}

   Zweidimensionale Arrays sind Pointer, die auf den Beginn eines
   Speicherbereichs zeigen, der Pointer zum entsprechenden Datentyp enthält.
   Diese wiederumg zeigen dann auf einen eigenen Block (der irgendwo liegen
   kann) der dann die primitiven Daten enthält.
   
   \subsection{Einarbeitung in \texttt{cmake}}
   
   \texttt{cmake} vereinfacht gegenüber \texttt{make} vor allem das Einbinden
   von Bibliotheken, die man in normalen Makefiles oft manuell suchen und
   verlinken muss. \texttt{cmake} kommt mit Skripten, die zum Auffinden vieler
   Bibliotheken für verschiedene Systeme dienen, und so den Kompilationsprozess
   relativ plattformunabhängig gestalten. Grundlegende Funktionalitäten sind die
   Eingabe von In- und Outputverzeichnissen, das Spezifizieren von Build-Targets
   und den dazugehörigen Bibliotheken, sowie das Kompilieren zu einer
   Bibliotheksdatei. Es ist auch einfach möglich, nicht-standard-artige
   Verezeichnisse zum Durchsuchen an \texttt{cmake} zu übergeben.

   Der Unterschied zwischen In-Source und Out-of-Source-Builds ist, dass für die
   letzteren sämtliche beim Build entstehenden Dateien an einen Ort außerhalb
   des Quellverzeichnisses abgelegt werden.

   \subsection{Hauptfenster mit \texttt{glut}}
   
   \lstinputlisting[caption={mainwindow.h}]{src/include/mainwindow.h}
   \lstinputlisting[caption={mainwindow.c}]{src/mainwindow.c}
   \lstinputlisting[caption={main.c}]{src/main.c}
   \lstinputlisting[caption={CMakeLists.txt},language=make]{src/CMakeLists.txt}
   
   
\end{document}

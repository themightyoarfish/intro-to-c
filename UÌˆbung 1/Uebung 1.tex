\documentclass{article}
\usepackage[utf8]{inputenc}
\title{Einführung in C++ -- Übung 1 \\ Testatgruppe A (Isaak)}
\author{Rasmus Diederichsen \and Simon Kern}
\date{\today}
\usepackage[ngerman]{babel}
\usepackage{microtype,
            textcomp,
            xcolor,
            listings,
            tikz,
            IEEEtrantools,
            array,
            amsmath,
            amssymb,
            graphicx,
            subcaption,
            lmodern}
\usepackage[pdftitle={Übung 1}, 
       pdfauthor={Rasmus Diederichsen}, 
       hyperfootnotes=true,
       colorlinks,
       bookmarksnumbered = true,
       linkcolor = lightgray,
       plainpages = false,
       citecolor = lightgray]{hyperref}
\usepackage[T1]{fontenc}
\renewcommand{\rmdefault}{ugm}
\usepackage[urw-garamond]{mathdesign}
\usepackage[all]{hypcap}
\renewcommand{\thesubsection}{Aufgabe \arabic{section}.\arabic{subsection}}

\lstset{
   basicstyle=\footnotesize\ttfamily,
   breaklines=true,
   commentstyle=\color{blue},
   keywordstyle=\color{purple}\textbf,
   numberstyle=\tiny\color{gray},
   numbers=left,
   stringstyle=\color{olive},
}

\begin{document}

   \maketitle

   \section{}
   \setcounter{subsection}{1}

   \subsection{}
   
   \lstinputlisting[language=c]{hello.c}
   \subsection{}
   
   \lstinputlisting[language=make]{makefile}
   \subsection{}
   
   \begin{itemize}
      \item Standardmäßig liegen Bibliotheken in \texttt{/usr[/local]/lib/}
         während sich die Header in \texttt{/usr[/local]/include/} befinden
      \item Der Präprozessor durchsucht standardmäßig alle Pfade, die im
         \texttt{C\_INCLUDE\_PATH} stehen, neben
         den obigen Standardverzeichnissen und einigen anderen.
      \item Für Bibliotheken ist die Variable \texttt{LIBRARY\_PATH} nützlich
      \item Um Umgebungsvariablen persisent zu setzen, bietet sich eine
         Anweisung der Form \texttt{export \$\emph{Variable} = \emph{Wert}} an,
         die in eine der Dateien geschrieben wird, die die Shell beim Start
         einliest (etwa \texttt{.profile} oder \texttt{.bash\_login} oder
         \texttt{.bashrc} für GNU Bash).
      \item \texttt{-I/\emph{dir}} weist den Präprozessor an, \emph{dir}
         ebenfalls nach Headern zu durchsuchen, \texttt{-L} hat dieselbe
         Funktion für Libraries. \texttt{-l\emph{foo}} Weist den Linker an,
         gegen die Bibliothek \texttt{foo} bzw. das Archiv \texttt{libfoo.a} zu
         linken.
      \item Die Syntax mit spitzen Klammern weist den Präprozessor an, nach der
         Header-Datei in den oben angegebenen Standardverzeichnissen sowie den
         durch die \texttt{-I}-Option spezifizierten zu suchen.
         Diese Syntax ist für System-Header gedacht.

         Schreibt man stattdessen Anführungszeichen, weist man auf
         selbstgeschriebene Header hin und es wird zunächst im
         Arbeitsverzeichnis, dann in den durch die \texttt{-iquote}-Option
         spezifizierten Verzeichnissen und zuletzt ebenfalls in den
         Standardverzeichnissen gesucht.
      \item Beim statischen Linken werden alle Bibliotheksfunktionen direkt in
         das entstehende Programm integriert, sodass diese nicht zwangsläufig
         auf dem System vorhanden sein müssen, auf dem es ausgeführt wird. Beim
         dynamischen Linken wird eine Bibliothek nicht direkt in die ausführbare
         Datei gesteckt, sondern diese bedient sich zur Laufzeit aus ihrem Code.
         Dies erfordert, dass die Bibliothek (\texttt{.so}, \texttt{.dll},
         \texttt{.dylib}) auf dem System vorhanden ist. Man kann Einfluss auf
         das Linken nehmen, indem man selbst angibt, ob gegen dynamische
         Libraries gelinkt werden soll. \texttt{ld} akzeptiert hierzu die
         Flags \texttt{-bstatic} und \texttt{-bdynamic} bzw. systemabhängige
         Varianten. Hierbei müssen die benötigten Libraries dann als Archive
         vorliegen, falls das Kompilieren gelingen soll.
      \item Mit \texttt{ldd} oder \texttt{otool} kann man herausfinden, welche
         dynamischen Bibliotheken von einer Datei (ausführbar, object file, \texttt{.a} oder
         \texttt{.so}) benötigt werden.
      \item \texttt{nm} zeigt die Symboltabelle der Bibliothek an.
      \item Man kann mit \texttt{dlopen} zur Laufzeit Bibliotheken laden. Wenn
         die Bibliothek auf dem eigenen System weder statisch noch dynamisch
         vorhanden ist, muss man vermutlich diesen Weg wählen.
      \item Bibliotheken stehen unter bestimmten Lizenzen, die beeinflussen, ob
         und wie man sie im eigenen Programm benutzen kann, will man dieses
         verteilen.
      \item Ein Paketmanager übernimmt die Aufgabe, Programme und Programmteile
         für ein System zu kompilieren, sodass der Benutzer nicht selbst dafür
         sorgen muss, dass alle Dependencies vorhanden sind und die Software
         richtig installiert wird. Man kann prinzipiell die gesamte Arbeit von
         Hand machen, wenn man eine hohe Schmerztoleranz hat.
   \end{itemize}


   \end{document}

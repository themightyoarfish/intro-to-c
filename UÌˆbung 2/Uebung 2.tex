\documentclass{article}
\usepackage[utf8]{inputenc}
\title{Einführung in C++ -- Übung 2 \\ Testatgruppe A (Isaak)}
\author{Rasmus Diederichsen \and Simon Kern}
\date{\today}
\usepackage[ngerman]{babel}
\usepackage{microtype,
            textcomp,
            xcolor,
            listings,
            tikz,
            IEEEtrantools,
            array,
            amsmath,
            amssymb,
            graphicx,
            subcaption,
            lmodern}
\usepackage[pdftitle={Übung 2}, 
       pdfauthor={Rasmus Diederichsen}, 
       hyperfootnotes=true,
       colorlinks,
       bookmarksnumbered = true,
       linkcolor = lightgray,
       plainpages = false,
       citecolor = lightgray]{hyperref}
\usepackage[T1]{fontenc}
\renewcommand{\rmdefault}{ugm}
\usepackage[urw-garamond]{mathdesign}
\usepackage[all]{hypcap}
\renewcommand{\thesection}{}
\renewcommand{\thesubsection}{Aufgabe \arabic{section}.\arabic{subsection}}

\lstset{
   basicstyle=\footnotesize\ttfamily,
   breaklines=true,
   commentstyle=\color{blue},
   keywordstyle=\color{purple}\textbf,
   numberstyle=\tiny\color{gray},
   numbers=left,
   stringstyle=\color{olive},
}

\begin{document}

   \maketitle

   \section{}
   \setcounter{section}{2}

   \subsection{}
   \lstinputlisting[language=C,caption={easter.c}]{EasterAlgorithm/easter.c}
   \lstinputlisting[language=make,caption={makefile for easter.c}]{EasterAlgorithm/makefile}

   \subsection{}
   \lstinputlisting[language=C,caption={sudoku.c}]{Sudoku/sudoku.c}
   \lstinputlisting[language=C,caption={sudoku.h}]{Sudoku/sudoku.h}
   \lstinputlisting[language=make,caption={makefile for sudoku.c}]{Sudoku/makefile}
   
   

   
\end{document}

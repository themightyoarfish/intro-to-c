\documentclass{article}
\usepackage[utf8]{inputenc}
\title{Einführung in C++ -- Übung 5 \\ Testatgruppe A (Isaak)}
\author{Rasmus Diederichsen}
\date{\today}
\usepackage[ngerman]{babel}
\usepackage{microtype, textcomp, xcolor, listings, tikz,
            IEEEtrantools, array, amsmath, amssymb, graphicx,
            subcaption, lmodern}
\usepackage[pdftitle={Übung 5}, 
       pdfauthor={Rasmus Diederichsen}, 
       hyperfootnotes=true,
       colorlinks,
       bookmarksnumbered = true,
       linkcolor = lightgray,
       plainpages = false,
       citecolor = lightgray]{hyperref}
\usepackage[T1]{fontenc}
\usepackage[explicit]{titlesec}
\renewcommand{\rmdefault}{ugm}
\usepackage[urw-garamond]{mathdesign}
\usepackage[all]{hypcap}
\titlespacing{\subsection}{0em}{\baselineskip}{1em}
\titleformat{\subsection}{\normalfont\Large\bfseries}{}{0em}{Aufgabe \arabic{section}.\arabic{subsection} #1}

\renewcommand{\thesection}{}

\lstset{
   basicstyle=\footnotesize\ttfamily,
   breaklines=true,
   commentstyle=\color{blue},
   keywordstyle=\color{purple}\textbf,
   numberstyle=\tiny\color{gray},
   numbers=left,
   language=C,
   stringstyle=\color{olive},
}

\begin{document}

   \maketitle

   \setcounter{section}{5}

   \subsection{Die Formeln}
   
   Hier gibt es nix zu erklären, da keine der beiden Formeln sinnhaft ist. 

   \subsection{Strukturen}
   
   \texttt{struct}s dienen in C dazu, mehrere Datenfelder in einem neuen
   Datentyp zusammenzufassen. 

   \subsection{Enumerations}
   
   \texttt{enum}s haben den Zweck, schnell eine große Anzahl Konstanten mit
   sinnvollen Bezeichnern einzuführen, die alle zur selben Kategorie gehören. Der Datentyp einer
   \texttt{enum}-Konstante ist ein Ganzzahltyp (vermutlich \texttt{int}).
   
   \lstinputlisting[caption={camera.c}]{src/camera.c}
   \lstinputlisting[caption={mainwindow.c}]{src/mainwindow.c}
\end{document}


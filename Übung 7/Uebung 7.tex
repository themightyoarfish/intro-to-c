\documentclass{article}
\usepackage[utf8]{inputenc}
\title{Einführung in C++ -- Übung 7 \\ Testatgruppe A (Isaak)}
\author{Rasmus Diederichsen}
\date{\today}
\usepackage[ngerman]{babel}
\usepackage{microtype, textcomp, xcolor, listings, tikz,
            IEEEtrantools, array, amsmath, amssymb, graphicx,
            subcaption, lmodern}
\usepackage[pdftitle={Übung 6}, 
       pdfauthor={Rasmus Diederichsen}, 
       hyperfootnotes=true,
       colorlinks,
       bookmarksnumbered = true,
       linkcolor = lightgray,
       plainpages = false,
       citecolor = lightgray]{hyperref}
\usepackage[T1]{fontenc}
\usepackage[explicit]{titlesec}
\renewcommand{\rmdefault}{ugm}
\usepackage[urw-garamond]{mathdesign}
\usepackage[all]{hypcap}
\titlespacing{\subsection}{0em}{\baselineskip}{1em}
\titleformat{\subsection}{\normalfont\Large\bfseries}{}{0em}{Aufgabe \arabic{section}.\arabic{subsection} #1}

\renewcommand{\thesection}{}

\lstset{
   basicstyle=\footnotesize\ttfamily,
   breaklines=true,
   commentstyle=\color{blue},
   keywordstyle=\color{purple}\textbf,
   numberstyle=\tiny\color{gray},
   numbers=left,
   language=C++,
   stringstyle=\color{olive},
  literate={ö}{{\"o}}1
           {ä}{{\"a}}1
           {ü}{{\"u}}1
}

\begin{document}

   \maketitle

   \setcounter{section}{7}

   \subsection{Implementierung von Quaternionen, Vektoren und Matrizen}

   \lstinputlisting[caption={Matrix.hpp}]{src/Matrix.hpp}
   \lstinputlisting[caption={Matrix.cpp}]{src/Matrix.cpp}
   \lstinputlisting[caption={Vertex.hpp}]{src/Vertex.hpp}
   \lstinputlisting[caption={Vertex.hpp}]{src/Vertex.hpp}
   \lstinputlisting[caption={Quaternion.cpp}]{src/Quaternion.cpp}
   \lstinputlisting[caption={Quaternion.cpp}]{src/Quaternion.cpp}
   
   \subsection{Zweidimensionaler Zugriff auf Matrizen}
   
   Wäre das Array zweidimensional, würde eine einfache Anwendung des
   \texttt{[]}-Operators eine Zeile, also einen Pointer des Arrays
   zurückliefern. Diese wäre dann wiederum mit \texttt{[]} indiziebar.  Für den
   Compiler ergäbe sich bei \texttt{m[x][y]} der Aufruf 
   \texttt{m.operator[](x)[y]}.  Allerdings wäre hierbei dann kein Bounds-Check
   möglich, weil \texttt{[][]} kein C++-Operator ist. Abhilfe könnte man
   schaffen, indem man eine dedizierte Klasse erstellt, deren einzige
   öffentliche Funktionalität \texttt{operator[](int i)} ist, die die Eingabe
   überprüfen kann, und unter Aufruf von \texttt{[]} aus der ersten Klasse den
   zweidimensionalen Zugriff realisiert.
   
   \begin{lstlisting}
   class Matrix {
      class Proxy {
         float* data;
         Proxy(float* f): data(f) {}
         float& operator[](int i) 
         {
            if(i >= 0 && i < 5)
               return data[i];
         }
      }

      float data[5][5];
      Proxy operator[](int i)
      {
         if(i >= 0 && i < 5)
         return Proxy(data[i]);
            
      }
   }
   \end{lstlisting}

\end{document}

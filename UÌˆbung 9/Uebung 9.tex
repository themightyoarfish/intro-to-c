\documentclass{article}
\usepackage[utf8]{inputenc}
\title{Einführung in C++ -- Übung 9 \\ Testatgruppe A (Isaak)}
\author{Rasmus Diederichsen}
\date{\today}
\usepackage[ngerman]{babel}
\usepackage{microtype, textcomp, xcolor, listings, tikz,
            IEEEtrantools, array, amsmath, amssymb, graphicx,
            subcaption, lmodern}
\usepackage[pdftitle={Übung 9}, 
       pdfauthor={Rasmus Diederichsen}, 
       hyperfootnotes=true,
       colorlinks,
       bookmarksnumbered = true,
       linkcolor = lightgray,
       plainpages = false,
       citecolor = lightgray]{hyperref}
\usepackage[T1]{fontenc}
\usepackage[explicit]{titlesec}
\renewcommand{\rmdefault}{ugm}
\usepackage[urw-garamond]{mathdesign}
\usepackage[all]{hypcap}
\titlespacing{\subsection}{0em}{\baselineskip}{1em}
\titleformat{\subsection}{\normalfont\Large\bfseries}{}{0em}{Aufgabe \arabic{section}.\arabic{subsection} #1}

\renewcommand{\thesection}{}

\lstset{
   basicstyle=\footnotesize\ttfamily,
   breaklines=true,
   commentstyle=\color{blue},
   keywordstyle=\color{purple}\textbf,
   numberstyle=\tiny\color{gray},
   numbers=left,
   language=C++,
   stringstyle=\color{olive},
   title=\lstname,
  literate={ö}{{\"o}}1
           {ä}{{\"a}}1
           {ü}{{\"u}}1
}

\begin{document}

   \maketitle

   \setcounter{section}{9}

   \subsection{Exception-Handling}
   
   Fehlerabfragen machen meines Erachtens nur beim Aufruf von
   \texttt{Vertex::normalize()} Sinn, und auch da nur beschränkt. Alle
   Arrayzugriffe durch den \texttt{[]}-operator geschehen durch Literale,
   weshalb hier ebenfalls kein \texttt{try}-Block nötig ist.

   \lstinputlisting{src/math/Vertex.cpp}
   \lstinputlisting{src/math/Quaternion.cpp}

   \subsection{Timestamps und Logging}
   
   \subsubsection*{Timestamp}
   \lstinputlisting{src/time/Timestamp.hpp}
   \lstinputlisting{src/time/Timestamp.cpp}

   \subsubsection*{Logger}
   \lstinputlisting{src/logging/Logger.hpp}
   \lstinputlisting{src/logging/Logger.cpp}

\end{document}
